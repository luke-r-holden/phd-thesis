\documentclass[12pt, twoside]{report}

\begin{document}

\noindent
The recombination line luminosity is given by
\begin{equation}
    L=n^2\alpha(T)Vfh\nu.
    \label{eq: l}
\end{equation}
Therefore, for a fixed volume $V$ of gas that has two density components ($n_1$ and $n_2$) of temperatures $T_1$ and $T_2$ and filling factors $f_1$ and $f_2$, the ratio of the luminosities for a single recombination line is
\begin{equation}
    \frac{L_1}{L_2}=\frac{\alpha(T_1)}{\alpha(T_2)}\frac{f_1}{f_2}\Bigg(\frac{n_1}{n_2}\Bigg)^2,
    \label{eq: l_ratio}
\end{equation}
or, assuming equal recombination coefficients,
\begin{equation}
    \frac{L_1}{L_2}=\frac{f_1}{f_2}\Bigg(\frac{n_1}{n_2}\Bigg)^2.
    \label{eq: l_ratio_equal_alpha}
\end{equation}

Now, consider that the mass of recombination-line emitting gas is given by
\begin{equation}
    M = nm_pVf,
    \label{eq: m}
\end{equation}
therefore the ratio of masses for the fixed volume of gas is
\begin{equation}
    \frac{M_1}{M_2}=\frac{n_1}{n_2}\frac{f_1}{f_2}
\end{equation}

For a fixed volume of gas with dense clumps and a tenuous extended component, such that the ratios of filling factors and densities are $f_1/f_2=10^3$ and $n_1/n_2=10^{-2}$, respectively, the luminosity ratio is
\begin{equation}
    \frac{L_1}{L_2}=10^3\times(10^{-2})^2=0.1,
\end{equation}
and the mass ratio
\begin{equation}
    \frac{M_1}{M_2}=10^{-2}\times10^3=10
\end{equation}
Therefore, a tenuous gas that is a hundred times less dense than (but has a filling factor a thousand times that of) dense clumps would have a mass that is 10 times that of the dense clumps, while only contributing 10\;per\;cent of the recombination line flux.

\end{document}