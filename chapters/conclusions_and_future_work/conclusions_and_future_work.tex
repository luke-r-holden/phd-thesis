\section{Thesis conclusions}
\label{chapter: conclusions_and_future_work: conclusions: introduction}

Outflows of gas accelerated by active galactic nuclei are routinely invoked by theoretical models of galaxy evolution to explain empirical scaling relations between supermassive black hole and host galaxy properties (e.g. \citealt{Silk1998, King2003, DiMatteo2005}), as well as the observed properties of the local galaxy population (e.g. \citealt{Schaye2015, Dave2019, Zinger2020}). In these models, prominent AGN-driven outflows heat and expel gas, suppressing the star formation rates of the host galaxies. 

Therefore, over the past 25 years, many observational studies have focused on comparing measured outflow properties to those required and predicted by the models (Figure\;\ref{fig: introduction: historical_context: galaxy_evolution: harrison2018_abstracts}; see \citealt{Harrison2018}). However, there are many uncertainties involved in deriving outflow properties from observations, and the validity of the comparisons to theoretical models is not clear. Principally, the true masses, mass outflow rates, kinetic powers, spatial extents, and acceleration mechanisms of AGN-driven outflows are highly uncertain --- this has limited robust interpretations regarding the role of outflows in galaxy evolution.

In this thesis, I have presented a series of detailed, multi-wavelength observational studies of four nearby active galaxies that address these sources of uncertainty. By developing, verifying, and using precise outflow-diagnostic techniques, I have precisely quantified key outflow properties and investigated their true natures and impacts on host galaxies. The results of these studies --- and their implications for the role of AGN in galaxy evolution --- are summarised and discussed here in the context of the major outstanding questions presented in Chapter\;\ref{chapter: introduction}.

\subsection{Which outflow phase(s) is dominant in terms of energetics, and how are the different
phases physically linked?}
\label{chapter: conclusions_and_future_work: conclusions: introduction: multiphase}

AGN-driven outflows are routinely observed in different gas phases (Table \ref{tab: introduction: outflows: energetics: multi-phase: outflow_phases}, Section\;\ref{section: introduction: outflows: energetics: multi-phase}; see \citealt{Cicone2018} for a review), including the coronal (e.g. \citealt{Riffel2013b, Speranza2022, VillarMartin2023, FonsecaFaria2023}), warm ionised (e.g. \citealt{VillarMartin1999, Holt2003, Zaurin2013, Harrison2014, Rose2018, Tadhunter2019, Revalski2021}), neutral atomic (e.g. \citealt{Morganti1998, Oosterloo2000, Rupke2005, Schulz2018, Su2023}), warm molecular (e.g. \citealt{Tadhunter2014, May2017}), and cold molecular (e.g \citealt{Alatalo2011, Cicone2014, Morganti2015, Oosterloo2017, RamosAlmeida2022, Audibert2023}). Several outflow phases have been observed to be simultaneously present in a small number of objects (e.g. \citealt{Morganti2005, Holt2011, Tadhunter2014, Riffel2015, Feruglio2015, Morganti2016, Finlez2018, Fluetsch2019}), and in some cases, these phases have been observed to be co-spatial and co-kinematic \citep{Morganti2007, Morganti2013_4c1250, Morganti2015, Tadhunter2014, Oosterloo2017, Rose2018}, indicating that they form part of the same outflow. However, due to the lack of objects for which there are robust outflow mass and kinetic power measurements of such multi-phase outflows, the relative importance of the different phases is unclear.

Using detailed, wide-wavelength-coverage, high-spatial-resolution optical/NIR observations of IC\;5063 --- a nearby Sey\;2 galaxy that presents prominent, multi-phase, jet-driven outflows --- I have precisely measured the mass outflow rates and kinetic powers of its warm ionised outflow phase for the first time (Chapter\;\ref{chapter: xshooter_ic5063}). Extensive previous multi-wavelength observations of the outflows in this object \citep{Morganti2005, Oosterloo2017, Venturi2021} allowed the energetics of the warm-ionised gas to be placed into an important multi-phase context: by recalculating the masses, mass outflow rates, and kinetic powers of the other observed phases with a consistent methodology, I showed that the neutral atomic outflow phase is dominant in terms of mass (which is three times that of the warm ionised and cold molecular phases combined) and kinetic power (which is two orders of magnitude higher than those of the other phases).

Similarly, the ULIRG/QSO2/GPS source F13451+1232 has been the focus of previous studies covering all gas phases. While compact ($r$\;\textless\;100\;pc) outflows had been previously detected and characterised in the primary nucleus of F13451+1232 in multiple gas phases \citep{Morganti2013_4c1250, Rose2018, Tadhunter2018, VillarMartin2023}, there had not been a robust detection of molecular outflows on any spatial scale (see discussion in \citealt{VillarMartin2023}). To address this and provide a complete multi-phase understanding of this important object, in Chapter\;\ref{chapter: alma_f13451_1232}, I presented and analysed high-spatial resolution ($0.113\times0.091$\;arcsecond or $247\times119$\;pc beam-size) ALMA CO(1--0) observations of the primary nucleus of F13451+1232, tracing the cold-molecular gas. High-velocity (400\;\textless\;$v$\;\textless\;680\;km\;s$^{-1}$), compact ($r$\;\textless\;120\;pc) emission close to the location of the primary nucleus is seen in the ALMA observations, which is interpreted as the cold-molecular component of the nuclear outflow. The mass outflow rate ($\dot{M}_\mathrm{out}\sim230$\;M$_\odot$) is more than an order of magnitude larger than those of the other phases, and the kinetic power ($\dot{E}_\mathrm{kin}\sim1.4$\;per\;cent of the AGN bolometric luminosity; $L_\mathrm{bol}=4.8\times10^{45}$\;erg\;s$^{-1}$: \citealt{Rose2018}) is several times that of the other phases combined. Crucially, including the cold-molecular outflow in calculating the \textit{total} multi-phase outflow kinetic power increases it to be in the range required by models of galaxy evolution, potentially indicating that it may play a role in the evolution of the merging galaxy system.

In all, the results of Chapters\;\ref{chapter: xshooter_ic5063} and \ref{chapter: alma_f13451_1232} directly show that the colder, non-ionised phases of AGN-driven outflows comprise the majority of the outflowing mass and kinetic power in the objects studied. This is a critical result: many observational studies of AGN-driven outflows only consider a single phase, and therefore may have significantly underestimated \textit{total} outflow masses and kinetic powers, potentially by orders of magnitude. Moreover, these chapters show that the warm ionised phase --- the most commonly observed outflow phase (e.g. \citealt{Kraemer2000II, VillarMartin1999, Das2006, Harrison2014, Tadhunter2019}) --- can only account for a small fraction of the overall outflowing gas.

In addition to determining multi-phase outflow energetics for two objects (see also Chapter\;\ref{chapter: stis_seyferts} for a discussion on the outflow phases in the Sey\;2 galaxy NGC\;1068), in Chapter\;\ref{chapter: xshooter_ic5063} I investigated how the different outflow phases are physically related. As noted earlier, previous studies had identified co-kinematic and co-spatial outflowing gas in different phases for a small number of objects, however, it was not clear why multiple phases are seen in a given outflow. This is particularly important to determine, as molecular gas is seen to be outflowing in some cases (e.g. \citealt{Tadhunter2014}): it would perhaps be expected that any outflow-driving mechanism would destroy the molecules, and so it is unclear how molecules can be accelerated to the high velocities observed. The high spatial resolution and wide wavelength coverage (from the UV to NIR) of the VLT/Xshooter observations allowed me to investigate the spatial flux distributions of various emission lines that are produced by different phases of outflowing gas at the NW radio lobe of IC\;5063 (where the highest-velocity outflows are observed). Combining this with emission-line-ratio diagnostic diagrams ([OIII](5007/4363) vs HeII$\lambda$4686/H$\mathrm{\beta}$: \citealt{VillarMartin1999}; [FeII]$\lambda$12570/Pa$\mathrm{\beta}$ vs H$_2\lambda$21218/Br$\mathrm{\gamma}$: \citealt{Larkin1998, Rodriguez-Ardila2005, Riffel2013a, Colina2015, Riffel2021}) that provide information regarding the ionisation/excitation mechanisms (shock- or photoionisation) of the outflowing and non-outflowing warm-ionised and warm-molecular gas, I proposed that the observed outflow phases in IC\;5063 represent a post-shock cooling sequence (Section\;\ref{section: xshooter_ic5063: discussion: mechanisms: physical_relation}, Figure\;\ref{fig: xshooter_ic5063: cooling_sequence_schematic}). In this scenario, quiescent gas in the galaxy passes through shocks produced by jet-ISM interactions, is heated to the highly ionised hot phase ($T$\;\textgreater\;$10^6$\;K), and subsequently cools through the warm ionised, neutral atomic, and warm molecular phases, eventually accumulating in cold-molecular gas. This provides an explanation for why different gas phases are seen to be co-kinematic and co-spatial in IC\;5063 (and other active galaxies), and how molecular gas can be accelerated to high velocities: the molecules are destroyed, and then reform post-acceleration.

Taken together, this thesis demonstrates that the cooler, non-ionised gas phases can dominate the energetics of AGN-driven outflows --- therefore, the multi-phase nature of AGN-driven outflows must be carefully considered in order to ensure that outflow energetics are not significantly underestimated. Furthermore, this work presents evidence in support of post-acceleration cooling being a (or the) physical link between the different outflow phases.

\newpage

\subsection{What are the true electron densities of the warm ionised outflow phase, and what is the impact on outflow masses and kinetic powers?}
\label{section: conclusions_and_future_work: conclusions: electron_densities}

The greatest source of uncertainty regarding the kinetic powers of the most commonly observed outflow phase --- the warm ionised --- is the electron density of the outflowing gas. Principally, this is because the most commonly used methods, which involve measuring the traditional [SII](6717/6731) or [OII](3726/3729) emission-line ratios, are only sensitive up to moderate values of density, leading to many studies estimating or assuming densities of $n_e$\;\textless\;$10^{3}$\;cm$^{-3}$ (e.g. \citealt{Liu2013, Harrison2014, Fiore2017, Mingozzi2019}). However, true electron densities may be much higher (potentially by orders of magnitude), as indicated by recent studies that make use of different techniques that are sensitive to a wider range of electron densities (up to $n_e\sim10^7$\;cm$^{-3}$: \citealt{Holt2011, Baron2019b, Revalski2021}). Considering that observationally-derived warm-ionised outflow masses (Equation\;\ref{eq: introduction: outflows: energetics: mout}), mass outflow rates (Equation\;\ref{eq: introduction: outflows: energetics: mout_rate}), and kinetic powers (Equation\;\ref{eq: introduction: outflows: energetics: ekin}) are inversely proportional to the measured or assumed electron density, this indicates that much of the past work that used the traditional [SII](6717/6731) and [OII](3726/3729) ratios (or assumed low values of electron density) could have overestimated outflow energetics by orders of magnitude. This may have had a significant impact on the interpretations made regarding the role of warm-ionised outflows in galaxy evolution.

In Chapters\;\ref{chapter: xshooter_ic5063} and \ref{chapter: stis_seyferts}, I made use of a technique involving the transauroral [SII]$(4068+4076)/(6717+6731)$ and [OII]$(3726+3729)/(7319+7331)$ emission-line ratios \citep{Holt2011, Rose2018, Santoro2020, Spence2018, Davies2020, Speranza2022, Speranza2024} --- which are sensitive to a wider range of electron density ($2.0$\;\textless\;log$_\mathrm{10}(n_e$\;[cm$^{-3}$])\;\textless\;5.5) than the commonly-used ratios --- to derive precise electron densities. Using high-spatial-resolution, wide-wavelength-coverage observations, I derived spatially-resolved electron densities with the transauroral-line technique and found them to be above the sensitivity limit of the commonly-used methods ($n_e=10^{3.2-3.4}$\;cm$^{-3}$ for IC\;5063: Chapter\;\ref{chapter: xshooter_ic5063}; $n_e=10^{4.1-4.7}$\;cm$^{-3}$ for NGC\;1068, and $n_e=10^{3.7-4.0}$\;cm$^{-3}$ for NGC\;4151: Chapter\;\ref{chapter: stis_seyferts}), in some cases by two orders of magnitude. For IC\;5063 (Chapter\;\ref{chapter: xshooter_ic5063}), I was able to measure the electron density of the same outflowing gas with both the transauroral-line technique and the traditional [SII](6717/6731) ratio, allowing me to directly show that the [SII](6717/6731) ratio underestimated the electron density by half-an-order of magnitude. These studies mark the first (and to date, only) time that the transauroral-line technique has been used to derive electron densities for spatially-resolved outflows, ensuring that the densities measured were for the same outflowing gas as those determined with the traditional [SII](6717/6731) and [OII](3726/3729) ratios.

\newpage

In the case of the prototypical Seyfert galaxies NGC\;1068 and NGC\;4151, the densities measured with the transauroral-line ratios were in agreement with those previously derived for these objects by detailed photoionisation modelling (\citealt{Revalski2021}; see also \citealt{Revalski2022}). Taken together, the results of these studies show that electron densities can be higher than the sensitivity limit of the traditional emission-line ratios, and that methods that are sensitive to a wider range of density values are required for robust measurements. The transauroral-line method is ideal for this, as it does not depend on a large number of emission lines nor require computationally-expensive modelling (such as in the \citealt{Revalski2021} approach).

Subsequently, the densities derived with the transauroral-line ratios for IC\;5063, NGC\;1068, and NGC\;4151 were used to derive energetics for the spatially-resolved warm-ionised outflows in these objects. The resulting mass outflow rates ($\dot{M}_\mathrm{out}=0.03$--0.18\;M$_\odot$ for IC\;5063; $\dot{M}_\mathrm{out}=0.6$--3.7\;M$_\odot$ for NGC\;1068; $\dot{M}_\mathrm{out}=3.4$--6.9\;M$_\odot$ for NGC\;1068) and kinetic powers ($\dot{E}_\mathrm{kin}$\;\textless\;$3.0\times10^{-3}$\;per\;cent of $L_\mathrm{bol}$ for IC\;5063; $\dot{E}_\mathrm{kin}$\;\textless\;$0.5$\;per\;cent of $L_\mathrm{bol}$ for NGC\;1068; $\dot{E}_\mathrm{kin}$\;\textless\;$1.0$\;per\;cent of $L_\mathrm{bol}$ for NGC\;4151) were modest in all cases. This demonstrates the importance of precise electron-density diagnostics: underestimating electron densities can lead to mass outflow rates and kinetic powers being overestimated by orders of magnitude. The many previous observational studies that likely underestimated electron densities concluded that the high kinetic powers they measured (\;\textgreater\;5\;per\;cent\; of $L_\mathrm{bol}$) implied that the outflows will have a significant impact on their host galaxies. Therefore, in showing that true outflow kinetic powers may be significantly lower, this work challenges the common interpretation that warm-ionised outflows play an important role in galaxy evolution.

Prior studies that made use of the transauroral-line technique had been spatially unresolved. This led to concerns that the lines involved are emitted by dense clumps of gas within outflowing cloud complexes, and therefore do not trace the same outflowing gas clouds as other key diagnostic lines such as [OIII]$4959,5007$ and H$\beta$, casting uncertainty on the validity of the transauroral lines as a measure of outflow density \citep{Sun2017, Rose2018, Spence2018}. Given that the VLT/Xshooter observations of IC\;5063 presented in Chapter\;\ref{chapter: xshooter_ic5063} are spatially resolved, I was able to demonstrate that the transauroral [SII] and [OII] lines are emitted in the same locations --- and with the same kinematics as --- other key diagnostic lines (namely [OIII]$4959,5007$ and H$\beta$), providing strong evidence that they are emitted by the same cloud systems and thus alleviating concerns regarding their use in observational outflow studies. Moreover, in Chapter\;\ref{chapter: stis_seyferts}, I argued that, since the transauroral [SII] and [OII] lines have critical densities that are closer to those of the [OIII]$\lambda\lambda$4959,5007 doublet (commonly used to determine outflow kinematics), they are a better representation of the outflowing gas than the traditional [SII]$\lambda\lambda$6717,6731 and [OII]$\lambda\lambda$3726,3729 lines.

Finally, while the impact of varying photoionisation properties on transauroral-line-derived densities had been established to be less than an order of magnitude \citep{Santoro2020}, the effect of shock-ionisation on derived densities was not clear. This was particularly important for the case of NGC\;4151, as evidence of shock-ionisation in the NLR of this object was found in Chapter\;\ref{chapter: stis_seyferts}. To address this, I quantified the impact of shock-ionisation of varying parameters on the derived electron densities (including the case in which photoionisation is erroneously assumed), and found this to be a factor of a few in total --- much less than the potential orders-of-magnitude error incurred by using techniques that are not sensitive to a wide range of densities.

Overall, the analyses and results of Chapters\;\ref{chapter: xshooter_ic5063} and \ref{chapter: stis_seyferts} demonstrate the importance of robust density diagnostics in observational studies of the warm ionised phase of AGN-driven outflows, and provide strong support for the use of the transauroral lines for this purpose.

\subsection{What are the spatial extents of AGN-driven outflows?}
\label{conclusions_and_future_work: conclusions: spatial_extents}

Theoretical models of galaxy evolution that invoke AGN-driven outflows as a feedback mechanism predict them to extend to galaxy-wide scales ($r$\;\textgreater\;5\;kpc: \citealt{Silk1998, DiMatteo2005, Curtis2016, Zubovas2023}). However, long-slit observations of active galaxies --- combined with a technique called spectroastrometry (in which the spatial centroid position of an emission line is measured as a function of velocity along the slit; \citealt{Carniani2015, VillarMartin2016, Santoro2020}) --- and direct HST imaging/spectroscopy \citep{Fischer2018, Tadhunter2018} have determined outflow radii to be less than a few kiloparsecs. This indicates that the impact of outflows is limited to the central regions of galaxies, in contradiction to the predictions of models.

Consistent with these results, by using high-spatial-resolution ($0.113\times0.091$\;arcsecond or $247\times119$\;pc beam-size) ALMA CO(1--0) observations of the primary nucleus of the ULIRG F13451+1232 --- an object that is representative of the situation considered in models of galaxy evolution --- I detected a cold-molecular outflow with a radius of $r$\;\textless\;120\;pc (in agreement with the radii of the other outflow phases in this object: \citealt{Morganti2013_4c1250, Tadhunter2018}). This demonstrates that multi-phase AGN-driven outflows --- even when accelerated by luminous radio sources and bolometrically-luminous quasars --- can be compact, and therefore that high-spatial-resolution observations are required to robustly characterise them.

However, in contrast to this, studies making use of ground-based IFU observations of the warm ionised outflow phase have claimed evidence for outflows having galaxy-wide spatial extents \citep{Fu2009, Westmoquette2012, Liu2014, McElroy2015, Wylezalek2017}. Crucially, these studies did not account for the effects of atmospheric seeing, which may have beam-smeared compact-outflow emission across the IFU field of view, artificially producing the appearance of galaxy-wide outflows. To directly address this, in Chapter\;\ref{chapter: muse_f13451_1232} I presented a ground-based IFU study of F13451+1232: by carefully accounting for the beam-smeared emission of the well-known compact outflows discussed earlier, no evidence for warm-ionised outflows on galaxy scales was found, in direct contradiction to the predictions of models.

Moreover, the IFU study of F13451+1232 presented in Chapter\;\ref{chapter: muse_f13451_1232} also demonstrates the importance of accounting for atmospheric seeing in ground-based IFU observations: the contribution from the beam-smeared compact-outflow emission to the line profiles of the extended emission was significant to (at minimum) a radial distance ($r\sim2.5$\;arcseconds) that was six times that of the HWHM of the seeing disk. If not accounted for, this may have been misinterpreted as genuine galaxy-wide outflows, significantly altering the interpretation of the impact of the AGN on the host galaxy. Furthermore, I demonstrated that failure to account for beam smearing can lead to mass outflow rates being overestimated by an order of magnitude, and kinetic powers being overestimated by two orders of magnitude.

Combined with the analyses presented in Chapters\;\ref{chapter: xshooter_ic5063} and \ref{chapter: stis_seyferts} (the observations and techniques of which traced gas in the central kiloparsecs of active galaxies), the compactness of the cold-molecular outflow presented in Chapter\;\ref{chapter: alma_f13451_1232} and the lack of warm-ionised galaxy-wide outflows determined in Chapter\;\ref{chapter: muse_f13451_1232} provide evidence that AGN-driven outflows are limited to the central regions of their host galaxies. This contradicts a prediction of galaxy-evolution models, and could be interpreted as AGN-driven outflows not having a significant impact on galaxy-wide scales.

\subsection{How do AGN accelerate outflows?}
\label{section: conclusions_and_future_work: acceleration_mechanisms}

Although AGN-driven outflows are now routinely observed in active galaxies of various types, it is not clear how they are accelerated (see \citealt{Wylezalek2018} for a review). Theoretical models can explain observed outflow kinematics in two general ways: via radiation pressure (either broad, radiatively-driven winds that shock the colder ISM: e.g. \citealt{Hopkins2010, Meena2021}, or `in situ': e.g. \citealt{Crenshaw2015, Revalski2018}) or via shocks induced by jet-ISM interactions (e.g. \citealt{Sutherland2007, Wagner2011, Mukherjee2016, Mukherjee2018}). However, in many cases, observed outflow and NLR properties can be explained by both of these mechanisms (e.g. \citealt{Ulvestad1981, Axon1998, Das2005, Das2006, Mukherjee2016, Mukherjee2018, Meena2021}), and \textit{both} may be present in a given galaxy, with their relative contributions to outflow acceleration being unclear. Determining how to observationally distinguish between these mechanisms, and clarifying their relative importance, is crucial: theoretical models can now predict specific outflow properties for a particular acceleration mechanism (e.g. \citealt{Richings2021, Meenakshi2022a, Meenakshi2022b}), allowing observational studies to directly verify their physical mechanisms and assumptions. Moreover, detailed understandings of outflow acceleration are required to inform the sub-grid physics of large-scale cosmological simulations (such as EAGLE: \citealt{Schaye2015}, HORIZON-AGN: \citealt{Dubois2016}, and SIMBA: \citealt{Dave2019}).

Typically, observational studies make use of gas ionisation or excitation conditions to infer information about outflow acceleration (e.g. \citealt{Mingozzi2019, Venturi2021, Revalski2021}): it is assumed that different ionisation/excitation mechanisms correspond to different acceleration mechanisms. However, the regions of photo- and shock-ionisation overlap considerably in the most commonly used diagnostic diagrams \citep{Dopita1995, Dopita1996, Allen2008, Ji2020}, and it is not clear if a direct relation between outflow ionisation/excitation and acceleration exists.

Chapter\;\ref{chapter: xshooter_ic5063} presents a detailed study of the Sey\;2 galaxy IC\;5063, for which the strong spatial and kinematic relation between the radio jet and outflows implies acceleration by jet-ISM interactions. Using the [OIII](5007/4363) vs He\;II$\lambda$4686/H$\beta$ diagnostic diagram \citep{VillarMartin1999} --- for which photo- and shock- ionisation are better separated than in commonly-used diagrams --- the warm-ionised outflowing and non-outflowing gas in IC\;5063 was found to be AGN-photoionised. While this may have been interpreted as the outflows being radiatively-driven, employing the NIR [FeII]$\lambda$12567/Pa$\beta$ and H$_2$(1--0)S(1)\;2.212\;$\mu$m/Br$\gamma$ diagram \citep{Larkin1998, Rodriguez-Ardila2005, Riffel2013a, Colina2015, Riffel2021} revealed that the outflowing warm-molecular gas had some contribution from shock-excitation. This is consistent with a scenario in which the gas in IC\;5063 is shocked by jet-ISM interactions, cools, and then is re-ionised by radiation from the AGN. Importantly, this demonstrates that photoionisation cannot be used as a proxy for radiative acceleration.

In Chapter\;\ref{chapter: stis_seyferts}, the [OIII](5007/4363) vs He\;II$\lambda$4686/H$\beta$ and [NeV]3426/[NeIII]3869 vs HeII/H$\mathrm{\beta}$ diagrams were used to show that the $r$\;\textless\;200\;pc NLR gas in NGC\;1068 is predominantly photoionised with matter-bounded geometry \citep{Binette1996}, while the NLR in NGC\;4151 presents evidence for shock ionisation and/or radiation-bounded photoionisation. Notably, in both cases, the outflows are limited to the spatial extents of the radio structures, similar to what is observed in IC\;5063 --- this indicates that, for all three objects, the outflows are accelerated by radio jets. Despite this, the ionisation conditions vary significantly, demonstrating that the relationship between outflow ionisation/excitation and acceleration is complex.

Similarly, the compact molecular outflow detected in the primary nucleus of the ULIRG/QSO F13451+1232 --- which has much higher radio and optical luminosities than the Seyfert galaxies discussed earlier --- is spatially offset along the direction of the small-scale radio jet, indicating that it is jet-driven (Chapter\;\ref{chapter: alma_f13451_1232}). In addition, spatially-extended ($r\sim440$\;pc), lower-velocity molecular gas is seen beyond the extent of the radio structure, which may constitute a radiatively-driven component.

Overall, this thesis shows that the link between outflow ionisation/excitation and acceleration is complex and that determining outflow acceleration mechanisms requires detailed multi-wavelength information. By taking this approach for the objects considered in this thesis --- each having different radio and bolometric luminosities --- I argue that the dominant acceleration mechanism is jet-ISM interactions in all cases.

\section{Future work}
\label{section: conclusions_and_future_work: future_work}

Further significant progress in understanding the natures of AGN-driven outflows and their role in galaxy evolution will require distinct, yet complementary approaches. This includes spatially-resolved, detailed, multi-wavelength studies of individual active galaxies (such as those presented in this thesis), large-scale statistical studies of AGN populations, and improved comparisons between observation and theory.

\subsection{Detailed studies of active galaxies}
\label{section: conclusions_and_future_work: future_work: detailed_studies}

Undertaking further detailed, multi-wavelength studies that make use of precise outflow diagnostics will permit robust outflow properties and acceleration mechanisms to be determined in active galaxies of all types, informing the nature of feedback in galaxies of different host and AGN properties. A particularly useful sample would be one consisting of nearby, non-quasar-like ($L_\mathrm{bol}$\;\textless\;$10^{45}$\;erg\;s$^{-1}$), intermediate-radio-luminosity (23\;\textless\;log$_\mathrm{10}$($L_\mathrm{1.4\;GHz}$\;[W\;Hz$^{-1}$])\;\textless\;25) active galaxies: the radio luminosities would be in the range for which prominent outflows would be expected \citep{Whittle1988, Mullaney2013}, while the non-quasar-like optical luminosities would mean that any observable signatures of jet-accelerated outflows would be less likely to be obscured by photo-ionisation/excitation. Therefore, such a sample would permit further robust determinations of ionisation/excitation and acceleration mechanisms, in addition to determining precise outflow properties --- this would allow the true impact of outflows to be determined for a larger sample of AGN than is studied here, thus establishing how representative the outflow properties of the objects studied in Chapters\;\ref{chapter: xshooter_ic5063} and \ref{chapter: stis_seyferts} are of the general population of such AGN.

Detailed studies that use high-spatial-resolution, wide-wavelength-coverage observations can also further develop and expand upon the diagnostic methods used in this thesis, such as improved discrimination between acceleration by jets and radiation pressure: as demonstrated in Chapters\;\ref{chapter: xshooter_ic5063}, \ref{chapter: stis_seyferts}, and \ref{chapter: alma_f13451_1232}, this requires detailed, multi-wavelength observations. Further advances in comprehensively determining the relative importance of different driving mechanisms will require investigating the origins of NLR shocks (i.e. jets or radiatively-driven winds) --- detailed investigations into gas (e.g. excitation/ionisation states, pre- and post-shock densities) and radio-emission (e.g. multi-frequency fluxes, spectral indices) properties will likely play a role in this. Moreover, these approaches can be used in further investigations of the physical link(s) between different outflow phases, such as the post-shock cooling sequence proposed in Chapter\;\ref{chapter: xshooter_ic5063}; clarifying this situation in particular will require observing nearby objects that host prominent jet-ISM interactions with high-spatial-resolution IR spectroscopy (for example with instruments such as VLT/ERIS and JWST/NIRSpec).

One of the key points made in this thesis is that multi-wavelength observations of active galaxies are needed to determine the total mass outflow rates and kinetic powers of multi-phase outflows. However, to date, there are only a few objects for which this has been done robustly (including IC\;5063, NGC\;1068, and F13451+1232: Chapters\;\ref{chapter: xshooter_ic5063}, \ref{chapter: stis_seyferts}, and \ref{chapter: alma_f13451_1232}, respectively). In addition to ongoing projects that are investigating the impact of multi-phase outflows in nearby quasars --- such as QSOFEED \citep{RamosAlmeida2022} and QFeedS \citep{Jarvis2019} ---  observing existing samples of other AGN types that already have well-characterised outflows in one or more phases will enable multi-phase measurements to be rapidly expanded to a higher number of objects. For example, for the QUADROS sample of ULIRGs \citep{Rose2018, Spence2018, Tadhunter2018, Tadhunter2019}, the energetics of warm-ionised outflows have been precisely quantified using densities derived with the transauroral-line technique. Therefore, observations at different wavelengths (but on the same spatial scales) would provide further insights into the dominant outflow phases and reveal total outflow properties.

Moreover, the crucial result determined throughout this thesis (and supported by the analyses of Chapters\;\ref{chapter: alma_f13451_1232} and \ref{chapter: muse_f13451_1232} in particular) --- that AGN-driven outflows appear to be limited to the central regions ($r$\;\textless\;5\;kpc) of host galaxies --- requires verification, principally to ensure that the situation in F13451+1232 is representative of the general AGN population. Further IFU (e.g. VLT/MUSE) observations of ULIRGs such as F13451+1232 --- which bear resemblance to the situation considered in models of galaxy evolution --- will be ideal for this. For example, using the beam-smearing-corrected techniques and analyses presented in Chapter\;\ref{chapter: muse_f13451_1232} with IFU observations of other ULIRGs in the QUADROS sample will allow true warm-ionised outflow extents to be determined for several objects that would be expected to host galaxy-wide outflows. Complementary to this, the observations used in previous ground-based IFU studies of QSOs that claimed evidence for galaxy-wide outflows (but did not account for beam smearing) can be revisited using the techniques developed in Chapter\;\ref{chapter: muse_f13451_1232}, thus providing a direct test of outflow spatial extents in objects of different types and further quantifying the impact of atmospheric seeing on observationally-derived outflow properties.

\subsection{Statistical studies of precise outflow properties}
\label{section: conclusions_and_future_work: future_work: statistical_studies}
\vspace*{-10pt}
Complementary to the approach taken in this thesis, future statistical studies of AGN-driven outflows will be required to determine how representative the results of detailed studies are, establish scaling relations between the properties of AGN / host galaxies and outflows, and determine the role of outflows in AGN-feedback at the population level. Previous large-scale statistical studies have been limited to determining outflow kinematics (e.g. with the [OIII] emission-line velocity width: \citealt{Mullaney2013}), but significant advances in our understanding will require outflow kinetic powers to be calculated for a large number of objects. Such an approach was attempted by \citet{Fiore2017} (see Figure\;\ref{fig: introduction: outflows: acceleration_mechanisms: fiore2017_mout_ekin_lbol}), however, the parameters used to derive the mass outflow rates and kinetic powers presented in that study were aggregated from various prior studies, each with different object-selection functions (hence different observational biases), and none of which accounted for the key uncertainties that this thesis demonstrates are important. Therefore, applying the techniques developed, verified and used in this thesis to spectroscopic survey data will permit --- for the first time --- statistical studies of precise outflow energetics with uniform methodology.

The improved sensitivity and wavelength coverages of the next generation of multi-object spectrographs such as the Dark Energy Spectroscopic Instrument (DESI; \citealt{Levi2019}) and the William Herschel Telescope (WHT) Enhanced Area Velocity Explorer (WEAVE: \citealt{Shoko2023}) will be ideal for this: they will enable statistical studies that make use of faint diagnostic lines (such as the transauroral [OII] and [SII] doublets) for the local AGN population, in addition to careful analysis of brighter spectral features (such as [OIII]$\lambda\lambda4959,5007$ and [SII]$\lambda\lambda6717,6731$; see discussions in Sections\;\ref{section: xshooter_ic5063: properties_of_outflowing_gas: uvb_vis_analysis_and_results: trad_densities} and \ref{section: muse_f13451_1232: analysis_and_results: extended_emission: aperture_properties}) for objects at higher redshift. Furthermore, the intermediate spectral resolutions of these instruments (60\;\textless\;${\Delta}V$\;\textless\;160\;km\;s$^{-1}$) will allow for outflowing and non-outflowing gas to be kinematically separated, permitting precise measurements of outflow properties in addition to gas ionisation mechanisms (which will provide indications of acceleration mechanisms for a large number of objects).

In particular, the WEAVE-LOFAR survey \citep{Smith2016, Shoko2023} will be ideal for this type of work: in addition to the key features outlined above, the survey will be radio-selected based on LOFAR (LOw Frequency ARray: \citealt{vanHaarlem2013}) low-frequency (10--240\;MHz) observations, meaning that previous results determined with optically-selected surveys (such as SDSS: \citealt{York2000}) can be verified for a different selection function based on low-frequency radio observations. In addition to observing a large number of HERGs (high excitation radio galaxies: \citealt{Laing1994, Tadhunter1998, Best2012}) --- objects similar in their observable properties to the galaxies considered in this thesis --- the majority of the expected AGN observed by WEAVE-LOFAR will be low excitation radio galaxies (LERGs). Despite being the dominant radio galaxy population in the local Universe \citep{Hardcastle2007}, relatively few studies of AGN-driven outflows in LERGs have been performed \citep{Ruffa2022}. Moreover, complementary low-frequency radio observations for all WEAVE-LOFAR sources will be available (at 6\;arcsecond resolution from the LoTSS survey: \citealt{Shimwell2017}), thus the variation of outflow properties in optical and radio AGN-subclasses can be investigated.

\subsection{Improved comparisons between observational and theoretical work}
\label{section: conclusions_and_future_work: future_work: improved_comparisons}

Since the advent of the field of AGN-driven outflows three decades ago, many observational studies have sought to verify the role of outflows in galaxy evolution by comparing observationally-derived outflow properties to the requirements and predictions of theoretical models. However, as discussed in Section\;\ref{section: introduction: outflows: comparisons_to_models}, in many cases it is not clear if such comparisons are valid, casting uncertainty regarding the interpretations made. Therefore, significant progress in the field of AGN-driven outflows will require consistent dialogue and collaboration between observers and theorists, which will be crucial to avoiding the misunderstanding and misinterpretation of results and improving interpretations. Principally, this dialogue has the potential to produce meaningful observables that can be used to verify the requirements and predictions of models, and the results of observational studies can inform the physical mechanisms included in the models themselves.

Such improved comparisons between theory and observation will likely involve detailed observations of individual objects and tailored simulations of those objects: for example, such as has been performed for IC\;5063 \citep{Morganti2015, Mukherjee2018} and the Teacup galaxy \citep{Audibert2023}. While \textit{exact} comparisons of values (such as coupling efficiencies and spatial extents) should be avoided due to the uncertainties and limitations inherent in both observationally-derived properties and modelling, these studies can test if the physics and assumptions of models are broadly valid.

Moreover, simulations of different outflow acceleration mechanisms can now predict a range of physical conditions and properties for gas in multiple phases (e.g. \citealt{Richings2021, Meenakshi2022a, Meenakshi2022b}) --- verifying these observationally with multi-wavelength observations and precise gas diagnostics provides an unprecedented opportunity to understand how outflows are accelerated, as well as their impact on host galaxies. The improved understanding of the underlying physical mechanisms can then be used to inform both further detailed modelling and the sub-grid physics of large-scale cosmological simulations; the predictions of the latter can be directly verified using the results of statistical studies made with spectroscopic surveys.

In conclusion, the precise diagnostic techniques presented in this thesis, applied to both detailed observations and large-scale statistical studies, in addition to improved dialogue between observers and theorists, will be crucial in making further advances in our understanding of AGN-driven outflows and their role in galaxy evolution.

\vfill

\section*{Chapter acknowledgements}
\addcontentsline{toc}{section}{\protect\numberline{}Chapter acknowledgements}

I thank Giovanna Speranza and Evgenii Chaikin --- my co-chairs of the `Improving comparisons between theoretical models and observations of AGN-driven outflows' breakout session at the `The importance of jet-induced feedback on galaxy scales' Lorentz Center workshop which informed the discussion in Section\;\ref{section: conclusions_and_future_work: future_work: improved_comparisons} of this chapter --- for their assistance in organising, hosting, and documenting the session; I also thank the attendees of the session for their insightful discussion. Furthermore, I thank Chris Harrison and Jarlath McKenna for their assistance in developing the ideas presented in Section\;\ref{section: conclusions_and_future_work: future_work: detailed_studies} of this chapter.