\vspace*{-24pt}

\textbf{Compact Steep Spectrum (CSS) source} = A luminous radio source ($L_\mathrm{1.4\;GHz}$\;\textgreater\;$10^{25}$\;W) with a steep spectral index ($\alpha$\;\textless\;$-0.5$): see \citet{ODea1998} and \citet{ODea2021} for reviews. \\

\noindent
\textbf{Coronal emission lines} = Warm-ionised (10,000\;\textless\;$T$\;\;\textless\;30,000\;K) emission lines arising from species with ionisation energies above $E_\mathrm{ion}$\;\textgreater\;100\;eV. \\

\noindent
\textbf{Emission-line ratio} = A ratio of the fluxes of two (or more) emission lines. \\

\noindent
\textbf{Galaxy-wide / galaxy scales} = Spatial scales similar to that of a typical AGN host galaxy; here, I define this as $r$\;\textgreater\;5\;kpc. \\

\noindent
\textbf{Gigahertz Peaked Spectrum (GPS) source} = A luminous radio source ($L_\mathrm{1.4\;GHz}$\;\textgreater\;$10^{25}$\;W\;Hz$^{-1}$) with a radio spectrum that peaks at gigahertz frequencies: see \citet{ODea1998} and \citet{ODea2021} for reviews. \\

\noindent
\textbf{High radio luminosities (radio-luminous)} = log$_\mathrm{10}(L_\mathrm{1.4\;GHz}$\;[W\;Hz$^{-1}$])\;\textgreater\;25. \\

\noindent
\textbf{Intermediate radio luminosities} = 23\;\textless\;log$_\mathrm{10}(L_\mathrm{1.4\;GHz}$\;[W\;Hz$^{-1}$])\;\textless\;25. \\

\noindent
\textbf{Matter-bounded geometry} = Nebular geometry in which the outer edge of the ionised region of a gas cloud is the outer edge of the cloud itself; the cloud is completely ionised, and lacks a partially-ionised or neutral-atomic zone. \\

\noindent
\textbf{Outflow} = Bulk motions of gas that are accelerated by AGN (e.g. via radiation pressure or shocks). \\

\noindent
\textbf{Outflow energetics} = Outflow masses (Equation \ref{eq: introduction: outflows: energetics: mout}), mass outflow rates (Equation \ref{eq: introduction: outflows: energetics: mout_rate}), and kinetic powers (Equation \ref{eq: introduction: outflows: energetics: ekin}); see also coupling efficiencies (Equation \ref{eq: introduction: outflows: introduction: e_f}). \\

\noindent
\textbf{Quasar} = A galaxy that shows AGN-like emission in its nucleus (line ratios consistent with the AGN regions of the \citealt{Baldwin1981} diagnostic diagrams, and line widths of FWHM\;\textgreater\;200\;km\;s$^{-1}$), with a bolometric luminosity of $L_\mathrm{bol}$\;\textgreater\;$10^{45}$\;erg\;s$^{-1}$. \\

\noindent
\textbf{Quiescent} = Non-outflowing; kinematically-quiescent. \\

\noindent
\textbf{Radiation-bounded geometry} = Nebular geometry in which the edge of the ionised region of a gas cloud is within the cloud itself, so that there exists an ionisation front; ionised, partially-ionised, and neutral-atomic gas is present within the cloud. \\

\noindent
\textbf{(Radio) jet} = A collimated stream of very fast ($v\sim0.1c$) plasma produced by magnetohydrodynamical processes in the central AGN engine. \\

\noindent
\textbf{Radio structure} = The observed structure of radio emission in an active galaxy, often consisting of a core and/or lobe(s). \\

\noindent
\textbf{Seyfert galaxy} = A galaxy that shows AGN-like emission in its nucleus (line ratios consistent with the AGN regions of the \citealt{Baldwin1981} diagnostic diagrams, and line widths of FWHM\;\textgreater\;200\;km\;s$^{-1}$), with a bolometric luminosity of $L_\mathrm{bol}$\;\textless\;$10^{45}$\;erg\;s$^{-1}$. \\


\noindent
\textbf{Ultra Luminous Infrared Galaxy (ULIRG)} = A galaxy with an infrared luminosity above $10^{12}$\;L$_\odot$ ($L_\mathrm{5-500\;\mu{m}}$\;\textgreater\;$3.8\times10^{45}$\;erg\;s$^{-1}$ \citealt{Sanders1996}). \\

\noindent
\textbf{Wind} = Very fast ($v$\;\textgreater\;10,000\;km\;s$^{-1}$) gas that has been accelerated by radiation pressure close to the accretion disk of an AGN (see \citealt{Hopkins2010}).
